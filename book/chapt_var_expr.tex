\chapter{Variables, Expressions and Statements}

One of the most powerful features of a programming language 
is the ability to manipulate {\bf variables}.  Broadly 
speaking, a variable is a name that refers to a value. It 
might be more accurate to say that a variable is a container 
that has a name and holds a value.
\index{variable}


\section{Assignment Statements}
\label{variables}
\index{assignment!statement}
\index{statement!assignment}
\index{$=$ assignment operator}
\index{operator!assignment}

An {\bf assignment statement} uses the equals sign {\tt =} 
and  gives a value to a variable, but, before you can assign 
a value to a variable, you first need to create the 
variable by declaring it (if it does not already 
exist):

\begin{verbatim}
> my $message;          # variable declaration, no value yet
> $message = 'And now for something completely different';
And now for something completely different
> my $number = 42;      # variable declaration and assignment
42
> $number = 17;         # new assignment
17
> my $phi = 1.618033988;
1.618033988
>
\end{verbatim}
%
\index{assignment}
\index{statement!assignment}
\index{declaration!variable}
\index{variable!declaration}

This example makes four assignment statements.  The first 
assigns a string to a new variable named {\tt \$message}, 
the second assigns the integer {\tt 42} to {\tt \$number}, the 
third reassigns the integer 17 to {\tt \$number}, and the fourth 
assigns the (approximate) value of the golden ratio to {\tt \$phi}.

There are two important syntax features to understand here.

First, in Perl, variable names start with a so-called 
\emph{sigil}, i.e., a special non-alphanumeric character such 
as \verb'$', \verb'@', \verb'%', \verb'&', and some others. 
This special character 
tells us and the Perl compiler (the program that reads 
the code of our program and transforms it into computer 
instructions) which kind of variable it is. For example, the 
\$ character indicates that the variables above are all 
\emph{scalar variables}, which means that they can contain 
only one value at any given time. We'll see later other 
types of variables that may contain more than one value.
\index{sigil}
\index{scalar}
\index{variable!scalar}
\index{scalar}

Second, notice that all three variables above are first 
introduced by the keyword {\tt my}, which is a way of 
declaring a new variable. Whenever you create a new 
variable in Perl, you need to \emph{declare} it, 
i.e., tell Perl that you're going to use that new 
variable; this is most commonly done with the 
{\tt my} keyword, which declares a \emph{lexical} variable. 
We will explain later what a lexical variable is; let's 
just say for the time being that it enables you to 
make your variable local to a limited part of your code. 
One of the good consequences of the requirement to declare 
variables before you use them is that, if you accidentally 
make a typo when writing a variable name, the compiler 
will usually be able to tell you that you are using a 
variable that has not been declared previously and 
thus help you find your error. This has other far-reaching 
implications which we will examine later.
\index{my}
\index{declaring variables}
\index{variable!declaration}
\index{lexical}
\index{variable!lexical}

When we wrote at the beginning of this section that a variable has 
to be declared before it is used (or just when it is used), it plainly 
means that the declaration has to be before (or at the point of) the 
variable's first use in the text file containing the program. We will see later that 
programs don't necessarily run from top to bottom in the order in which 
the lines or code appear in the program file; still, the variable 
declaration must be before its use in the text file containing the program.

If you neglect to declare a variable, you get a syntax error:
\index{syntax error}

\begin{verbatim}
> $number = 5;
===SORRY!=== Error while compiling <unknown file>
Variable '$number' is not declared
at <unknown file>:1
------> <BOL><HERE>$number = 5;
>
\end{verbatim}
%
Please remember that you may obtain slightly different 
error messages depending on the version of Rakudo you 
run. The above message was obtained in February 2016; 
with a newer version (October 2016), the same error 
is now displayed somewhat more cleanly as:
\begin{verbatim}
>
> $number = 5;
===SORRY!=== Error while compiling:
Variable '$number' is not declared
at line 2
------> <BOL><HERE>$number = 5;
>
\end{verbatim}

\index{state diagram}
\index{diagram!state}

A common way to represent variables on paper is to write the name with
an arrow pointing to its value.  This kind of figure is
called a {\bf state diagram} because it shows what state each of the
variables is in (think of it as the variable's state of mind).
Figure~\ref{fig.state2} shows the result of the previous example.

\begin{figure}
\centerline
{\includegraphics[scale=0.6]{figs/test_5.png}}
\caption{State diagram.}
\label{fig.state2}
\end{figure}



\section{Variable Names}
\index{variable}

Programmers generally choose names for their variables that
are meaningful---they document what the variable is used for.

Variable names can be as long as you like.  They can contain
both letters and numbers, but user-defined variable names 
can't begin with a number. Variable names are case-sensitive, 
i.e., {\tt \$message} is not the same variable as {\tt \$Message} 
or {\tt \$MESSAGE}. It is legal to use uppercase letters, but 
it is conventional to use only lower case for most variables 
names. Some people nonetheless like to use {\tt \$TitleCase} 
for their variables or even pure {\tt \$UPPERCASE} for 
some special variables.
\index{lower case}
\index{upper case}
\index{title case}
\index{case!lower}
\index{case!upper}
\index{case!title}


\index{Unicode}
\index{ASCII}
Unlike most other programming languages, Perl~6 does not require 
the letters and digits used in variable names to be plain ASCII. 
You can use all kinds of Unicode letters, i.e., 
letters from almost any language in the world, so that, for example,  
{\tt \$brücke}, {\tt \$payé} or {\tt \$niño} are 
valid variable names, which can be useful for non-English 
programmers (provided that these Unicode characters are 
handled correctly by your text editor and your 
screen configuration). Similarly, instead of using 
\verb"$phi" for the name of the golden ratio variable, 
we might have used the \emph{Greek small letter phi}, \verb'φ' 
(Unicode code point U+03C6), just as we could have used 
the \emph{Greek small letter pi}, $\pi$,  for the well-known 
circle circumference to diameter ratio:
\index{golden ratio}
\index{Unicode}
\index{phi}
\index{pi}

\begin{verbatim}
> my $φ = (5 ** .5 + 1)/2;       # golden ratio
1.61803398874989
> say 'Variable $φ = ', $φ;
variable $φ = 1.61803398874989
> my $π = 4 * atan 1; 
3.14159265358979
> # you could also use the pi or π built-in constant:
> pi
3.14159265358979
\end{verbatim}

The underscore character, \verb"_", can appear anywhere in a variable name.
It is often used in names with multiple words, such as 
\verb"$your_name" or \verb"$airspeed_of_unladen_swallow". 
\index{underscore character}

You may even use dashes to create so-called 
``kebab case''\footnote{Because the 
words appear to be skewered like pieces of food prepared for 
a barbecue.} and name those variables 
\verb"$your-name" or \verb"$airspeed-of-unladen-swallow", and this 
might make them slightly easier to read: a dash \verb'-' is 
valid in variable names provided it is immediately followed by an 
alphabetical character and preceded by an alphanumerical character. 
For example, \verb"$double-click" or \verb"$la-niña" are legitimate 
variable names. Similarly, you can use an apostrophe \verb"'" (or single 
quote) between letters, so \verb"$isn't" or \verb"$o'brien's-age" 
are valid identifiers.
\index{dash}
\index{apostrophe}
\index{single quote}
\index{quote!single}
\index{kebab case}
\index{case!kebab}


If you give a variable an illegal name, you get a syntax error:
\index{syntax error}

\begin{verbatim}
> my $76trombones = 'big parade'
===SORRY!=== Error while compiling <unknown file>
Cannot declare a numeric variable
at <unknown file>:1
------> my $76<HERE>trombones = "big parade";
>
> my $more§ = 100000;
===SORRY!=== Error while compiling <unknown file>
Bogus postfix
at <unknown file>:1
------> my $more<HERE>§ = 100000;
(...)
\end{verbatim}
%
{\tt \$76trombones} is illegal because it begins with a number.
{\tt \$more§} is illegal because it contains an illegal character, {\tt
§}. 

If you've ever used another programming language and stumbled 
across a terse message such as {\tt"SyntaxError: invalid syntax"}, 
you will notice that the Perl designers have made quite a bit 
of effort to provide detailed, useful,  and meaningful error 
messages. 
\index{error message}

Many programming languages have \emph{keywords} or \emph{reserved 
words} that are part of the syntax, such as {\tt if}, {\tt while}, 
or {\tt for}, and thus cannot be used for identifying variables 
because this would create ambiguity. There is no such problem 
in Perl: since variable names start with a sigil, the compiler 
is always able to tell the difference between a keyword and a 
variable. Names such as {\tt \$if} or {\tt \$while} are 
syntactically valid variable identifiers in Perl (whether 
such names make sense is a different matter).
\index{sigil}
\index{keyword}
\index{reserved word}



\section{Expressions and Statements}
\label{expr_and_statements}

An {\bf expression} is a combination of terms and operators.
Terms may be variables or literals, i.e., constant values such as a number or a string. A value all by itself is considered an expression, and so is
a variable, so the following are all legal expressions:
\index{expression}
\index{term}
\index{literal}

\begin{verbatim}
> 42
42
> my $n = 17;
17
> $n;
17
> $n + 25;
42
>
\end{verbatim}
%
When you type an expression at the prompt, the interpreter
{\bf evaluates} it, which means that it finds the value of
the expression.
In this example, {\tt \$n} has the value 17 and
{\tt \$n + 25} has the value 42.
\index{evaluate}

A {\bf statement} is a unit of code that has an effect, like
creating a variable or displaying a value, and usually needs to end 
with a semi-colon {\tt ;} (but the semi-colon can sometimes be omitted 
as we will see later):
\index{statement}
\index{semi-colon}

\begin{verbatim}
> my $n = 17;
17
> say $n;
17
\end{verbatim}
%

The first line is an assignment statement that gives a value to
{\tt \$n}.  The second line is a print statement that displays the
value of {\tt \$n}.

When you type a statement and then press {\tt Enter}, the 
interpreter {\bf executes} it, which means that it does 
whatever the statement says.
\index{execute}

An assignment can be combined with expressions using arithmetic 
operators. For example, you might write:
\index{operator}

\begin{verbatim}
> my $answer = 17 + 25;
42
> say $answer;
42
\end{verbatim}
%

The \verb'+' symbol is obviously the addition operator 
and, after the assignment statement, the \verb'$answer' 
variable contains the result 
of the addition. The terms on each side of the operator 
(here 17 and 25) are sometimes called the \emph{operands} 
of the operation (an addition in this case).
\index{operand}
\index{addition operator}
\index{assignment}

Note that the REPL actually displays the result of the 
assignment (the first line with ``42''), so that the 
print statement was not really necessary in this 
example \emph{under the REPL}; from now on, for the sake of 
brevity, we will generally omit the print statements in the 
examples where the REPL displays the result.
\index{REPL}

In some cases, you want to add something to a variable 
and assign the result to that same variable. This could 
be written:

\begin{verbatim}
> my $answer = 17;
17
> $answer = $answer + 25;
42
\end{verbatim}
%

Here, \verb"$answer" is first declared with a value of 17. The next 
statement assigns to \verb"$answer" the current value of 
\verb"$answer" (i.e., 17) + 25. This is such a common operation 
that Perl, like many other programming languages, has a 
shortcut for this:

\begin{verbatim}
> my $answer = 17;
17
> $answer += 25;
42
\end{verbatim}
%

\index{$+=$ augmented assignment operator}
The \verb"+=" operator combines the arithmetic addition operator 
and the assignment operator to modify a value and apply the result 
to a variable in one go, so that \verb"$n += 2" means: take 
the current value of \verb"$n", add 2, and assign the result to 
\verb"$n". This syntax works with all other arithmetic operators. 
For example, \verb"-=" similarly performs a subtraction and an 
assignment, \verb"*=" a multiplication and an assignment, etc. It 
can even be used with operators other than arithmetic operators, 
such as the string concatenation operator that we will see later.

Adding 1 to a variable is a very common version of this, so that 
there is a shortcut to the shortcut, the \emph{increment} operator, 
which increments its argument by one, and returns the incremented value:
\index{increment operator}
\index{++ increment operator}
\index{operator!++ (increment)}

\begin{verbatim}
> my $n = 17;
17
> ++$n;
18
> say $n;
18
\end{verbatim}
%
This is called the prefix increment operator, because the \verb"++" 
operator is placed before the variable to be incremented. There is 
also a postfix version, \verb"$n++", which first returns the current 
value and then increments the variable by one. It would not make 
a difference in the code snippet above, but the result can be very different 
in slightly more complex expressions. 

There is also a decrement operator \verb"--", which decrements its 
argument by one and also exists in a prefix and a postfix form. 
\index{decrement operator}
\index{\verb'--' decrement operator}
\index{operator!$--$ (decrement)}



\section{Script Mode}

So far we have run Perl in {\bf interactive mode}, which
means that you interact directly with the interpreter (the 
REPL). Interactive mode is a good way to get started,
but if you are working with more than a few lines of code, 
it can be clumsy and even tedious.
\index{interactive mode}

The alternative is to use a text editor and save code in a file 
called a {\bf script} and then run the interpreter in {\bf script mode} 
to execute the script.  By convention, Perl~6 scripts have names that 
end with {\tt .pl}, {\tt .p6} or {\tt .pl6}.
\index{script}
\index{script mode}

Please make sure that you're really using a \emph{text editor} 
and not a \emph{word-processing program} (such as MS Word, 
OpenOffice or LibreOffice Writer). There is a very large 
number of text editors available for free. On Linux, you might use 
\emph{vi} (or \emph{vim}), \emph{emacs}, \emph{gEdit}, or 
\emph{nano}. On Windows, you may use \emph{notepad} (very limited) 
or \emph{notepad++}. There are also many cross-platform editors  
or integrated development environments (IDEs) providing a 
text editor functionality, including \emph{padre}, \emph{eclipse}, 
or \emph{atom}. Many of these provide various syntax highlighting 
capabilities, which might help you use correct syntax (and 
find some syntax errors).
\index{syntax!highlighting}
\index{text editor!emacs}
\index{text editor!vi}
\index{text editor!vim}
\index{text editor!gEdit}
\index{text editor!padre}
\index{text editor!eclipse}
\index{text editor!nano}
\index{text editor!notepad++}
\index{text editor!atom}

Once you've saved your code into a file (say, for example, 
\verb'my_script.pl6'), you can run the program by issuing 
the following command at the operating system prompt (for example 
in a Linux console or in a \verb'cmd' window under Windows):
\begin{verbatim}
perl my_script.pl6
\end{verbatim}

Because Perl provides both modes,
you can test bits of code in interactive mode before you put them
in a script.  But there are differences between interactive mode
and script mode that can be confusing.
\index{interactive mode}
\index{script mode}

For example, if you are using the Perl~6 interpreter as a 
calculator, you might type:

\begin{verbatim}
> my $miles = 26.2;
26.2
> $miles * 1.61;
42.182
\end{verbatim}

The first line assigns a value to {\tt \$miles} and displays that value.  
The second line is an expression, so the
interpreter evaluates it and displays the result.  It turns out that a
marathon is about 42~kilometers.

But if you type the same code into a script and run it, you get no
output at all.  In script mode, an expression, all by itself, has no
visible effect.  Perl actually evaluates the expression, but it doesn't
display the value unless you tell it to:

\begin{verbatim}
my $miles = 26.2;
say $miles * 1.61;
\end{verbatim}

This behavior can be confusing at first. Let's examine why.

A script usually contains a sequence of statements.  If there
is more than one statement, the results appear one at a time
as the print statements execute.

For example, consider the following script:

\begin{verbatim}
say 1;
my $x = 2;
say $x;
\end{verbatim}
%
It produces the following output:

\begin{verbatim}
1
2
\end{verbatim}
%
The assignment statement produces no output.

To check your understanding, type the following statements in the
Perl interpreter and see what they do:

\begin{verbatim}
5;
my $x = 5;
$x + 1;
\end{verbatim}

Now put the same statements in a script and run it.  What
is the output?  Modify the script by transforming each
expression into a print statement and then run it again.

\section{One-Liner Mode}

Perl also has a \emph{one-liner mode}, which enables you 
to type directly a very short script at the operating 
system prompt. Under Windows, it might look like this:
\index{one-liner mode}
\label{one-liner mode}

\begin{verbatim}
C:\Users\Laurent>perl6 -e "my $value = 42; say 'The answer is ', $value;"
The answer is 42

\end{verbatim}

The {\tt -e} option tells the compiler that the script to 
be run is not saved in a file but instead typed at the 
prompt between quotation marks immediately after 
this option.

Under Unix and Linux, you would replace double quotation 
marks with apostrophes (or single quotes) 
and apostrophes with double quotation marks:
\index{apostrophe}
\index{quote mark}

\begin{verbatim}
$  perl6 -e 'my $value = 42; say "The answer is $value";'
The answer is 42

\end{verbatim}

The one-liner above may not seem to be very useful, but 
throwaway one-liners can be very practical to perform 
simple one-off operations, such as quickly modifying 
a file not properly formatted, without having to save a script 
in a separate file before running it.

We will not give any additional details about the one-liner 
mode here, but will give some more useful examples 
later in this book; for example, 
Subsection~\ref{one-liner-example},
Subsection~\ref{rot13_oneliner} (solving the ``rot-13'' exercise), or
Subsection~\ref{sol_cartalk} (solving the exercise on 
consecutive double letters). 



\section{Order of Operations}
\index{order of operations}
\index{PEMDAS}
\index{operator precedence}
\index{precedence!operator}

When an expression contains more than one operator, the order of
evaluation depends on the {\bf order of operations} or \emph{operator precedence}. 
For mathematical operators, Perl follows mathematical convention.
The acronym {\bf PEMDAS}\footnote{US students are sometimes taught 
to use the "Please Excuse My Dear Aunt Sally" mnemonics to remember 
the right order of the letters in the acronym.} is a useful 
way to remember the rules:

\begin{itemize}

\item {\bf P}arentheses have the highest (or tightest) precedence and can be used 
to force an expression to evaluate in the order you want. Since
expressions in parentheses are evaluated first, {\tt 2 * (3-1)} is 4,
and {\tt (1+1)**(5-2)} is 8. You can also use parentheses to make an
expression easier to read, as in {\tt (\$minute * 100) / 60}, even
if it doesn't change the result.
\index{$()$ parenthesis operator}

\item {\bf E}xponentiation has the next highest precedence, so
{\tt 1 + 2**3} is 9 (1 + 8), not 27, and {\tt 2 * 3**2} is 18, not 36.
\index{$**$ exponentiation operator}
\index{operator!$**$ (exponentiation)}

\item {\bf M}ultiplication and {\bf D}ivision have higher precedence
  than {\bf A}ddition and {\bf S}ubtraction.  So {\tt 2*3-1} is 5, not
  4, and {\tt 6+4/2} is 8, not 5.
\index{$*$ multiplication operator}
\index{operator!$*$ (multiplication)}
\index{$/$ division operator}
\index{operator!$/$ (division)}

\item Operators with the same precedence are usually evaluated 
from left to right (except exponentiation).  So in the expression 
{\tt \$degrees / 2 * pi}, the division happens first and the 
result is multiplied by {\tt pi}, which is not the expected 
result. (Note that {\tt pi} is not a variable, but a predefined 
constant in Perl~6, and therefore does not require a sigil.)  To 
divide by $2 \pi$, you can use parentheses:
\index{sigil}
\index{pi}
  
\begin{verbatim}
my $result = $degrees / (2 * pi);  
\end{verbatim}  
 
or write
  {\tt \$degrees / 2 / pi} or {\tt \$degrees / } $2 / \pi$, which 
  will divide \verb'$degrees' by 2, and then divide the result of 
  that operation by $\pi$ (which is equivalent \verb'$degrees' by 
   $2 \pi$.

\end{itemize}

I don't work very hard to remember the precedence of
operators.  If I can't tell by looking at the expression, I use
parentheses to make it obvious. If I don't know for sure which of two operators 
has the higher precedence, then the next person reading or maintaining 
my code may also not know.
\index{precedence}
\index{parentheses}


\section{String Operations}
\label{string_operations}
\index{string!operation}
\index{operator!string}
\index{coercion}
\index{type!coercion}

In general, you can't perform mathematical operations on strings, unless
the strings look so much like numbers that Perl can transform or \emph{coerce} them into numbers and still make sense, so the 
following are illegal:

\begin{verbatim}
'2'-'1a'    'eggs'/'easy'    'third'*'a charm'
\end{verbatim}
%

For example, this produces an error:

\begin{verbatim}
> '2'-'1a'
Cannot convert string to number: trailing characters after number 
in '1?a' (indicated by ?)
  in block <unit> at <unknown file>:1
\end{verbatim}
%
  
But the following expressions are valid because these strings 
can be coerced to numbers without any ambiguity:
\begin{verbatim}
> '2'-'1'
1
> '3'/'4'
0.75
\end{verbatim}
%

The \verb'~' operator performs {\bf string concatenation}, which means
it joins the strings by linking them end-to-end.  For example:
\index{string concatenation}
\index{string!concatenation}

\begin{verbatim}
> my $first = 'throat'
throat
> my $second = 'warbler'
warbler
> $first ~ $second
throatwarbler
\end{verbatim}
%
The {\tt x} operator also works on strings; it performs repetition.
For example:
\index{string repetition}

\begin{verbatim}
> 'ab' x 3;
ababab
> 42 x 3
424242
> 3 x 42
333333333333333333333333333333333333333333
\end{verbatim}

Notice that, although the {\tt x} operator somewhat looks like 
the multiplication operator when we write it by hand, {\tt x} 
is obviously not commutative, contrary to the {\tt *} 
multiplication operator. The first operator is a string or is 
\emph{coerced} to a string (i.e., transformed into a string: 
{\tt 42} is coerced to {\tt '42'}), and the second operator 
has to be a number or something that can be transformed 
into a number.
\index{commutativity}
\index{coercion}


\section{Comments}
\index{comment}

As programs get bigger and more complicated, they get more difficult
to read.  Formal languages are dense, and it is often difficult to
look at a piece of code and figure out what it is doing, or why.

For this reason, it is a good idea to add notes to your programs to explain
in natural language what the program is doing.  These notes are called
{\bf comments}, and they start with the \verb"#" symbol:

\begin{verbatim}
# compute the percentage of the hour that has elapsed
my $percentage = ($minute * 100) / 60;
\end{verbatim}
%
In this case, the comment appears on a line by itself.  You can also put
comments at the end of a line:

\begin{verbatim}
$percentage = ($minute * 100) / 60;     # percentage of an hour
\end{verbatim}
%
Everything from the {\tt \#} to the end of the line is ignored---it
has no effect on the execution of the program.

Comments are most useful when they document nonobvious features of
the code.  It is reasonable to assume that the reader can figure out
{\em what} the code does; it is more useful to explain {\em why}.

This comment is redundant with the code and useless:

\begin{verbatim}
my $value = 5;        # assign 5 to $value
\end{verbatim}
%
This comment, by contrast, contains useful information that 
is not in the code:

\begin{verbatim}
my $velocity = 5;     # velocity in meters/second. 
\end{verbatim}
%
Good variable names can reduce the need for comments, but
long names can make complex expressions hard to read, so there is
a tradeoff.
\index{variable name}


\section{Debugging}
\index{debugging}
\index{bug}

Three kinds of errors can occur in a program: syntax errors, runtime 
errors, and semantic errors.  It is useful
to distinguish between them in order to track them down more quickly.

\begin{description}

\item[Syntax error] ``Syntax'' refers to the structure of a program
  and the rules about that structure.  For example, parentheses have
  to come in matching pairs, so {\tt (1 + 2)} is legal, but 
{\tt 8)} is a \emph{syntax error}. \footnote{We are using 
``syntax error'' here as a quasi-synonym for ``compile-time error''; 
they are not exactly the same thing (you may in theory have 
syntax errors that are not compile-time errors and the other way 
around), but they can be deemed to be the same for 
practical purposes here. In Perl~6, compile-time errors 
have the ``===SORRY!==='' string at the beginning of 
the error message.}

\index{error!syntax}
\index{error message}
\index{syntax} 
\index{syntax!error}

If there is a syntax error
anywhere in your program, Perl displays an error message and quits 
without even starting to run your program, and you will 
obviously not be able to run the program.  During the first few
weeks of your programming career, you might spend a lot of
time tracking down syntax errors.  As you gain experience, you will
make fewer errors and find them faster.


\item[Runtime error] The second type of error is a runtime error, so
  called because the error does not appear until after the program has
  started running.  These errors are also called \emph{exceptions}
  because they usually indicate that something exceptional (and bad)
  has happened.  \index{runtime error} \index{error!runtime}
  \index{exception} \index{safe language} \index{language!safe}

Runtime errors are rare in the simple programs you will see in the
first few chapters, so it might be a while before you encounter one. 
We have seen one example of such errors, though, at the beginning 
of Section~\ref{string_operations} (p.~\pageref{string_operations}), 
when we tried to subtract \verb"'2'-'1a'". 


\item[Semantic error] The third type of error is \emph{semantic}, which
  means related to meaning.  If there is a semantic error in your
  program, it will run without generating error messages, but it will
  not do the right thing.  It will do something else.  Specifically,
  it will do what you \emph{told} it to do, but not what you 
  \emph{intended} it to do.
  \index{semantic error}
  \index{error!semantic} \index{error message}

Identifying semantic errors can be tricky because it requires you to work
backward by looking at the output of the program and trying to figure
out what it is doing.

\end{description}


\section{Glossary}

\begin{description}

\item[Variable]  Informally, a name that refers to a value. More 
accurately, a variable is a container that has a name and holds a value.
\index{variable}

\item[Assignment]  A statement that assigns a value to a variable.
\index{assignment}

\item[State diagram]  A graphical representation of a set of variables and the
values they refer to.
\index{state diagram}

\item[Keyword]  A reserved word that is used to parse a
program; in many languages, you cannot use keywords like 
{\tt if}, {\tt  for}, and {\tt while} as variable names. 
This problem usually does not occur in Perl because variable
names begin with a \emph{sigil}.
\index{keyword}
\index{sigil}

\item[Operand]  One of the values on which an operator operates.
\index{operand}

\item[Term]  A variable or a literal value.
\index{term}

\item[Expression]  A combination of operators and terms that
represents a single result.
\index{expression}

\item[Evaluate]  To simplify an expression by performing the operations
in order to yield a single value.
\index{evaluate}

\item[Statement]  A section of code that represents a command or action.  So
far, the statements we have seen are assignments and print statements. Statements usually end with a semi-colon.
\index{statement}

\item[Execute]  To run a statement and do what it says.
\index{execute}

\item[interactive mode (or interpreter mode):] A way of using the Perl 
interpreter by typing code at the prompt.
\index{interactive mode}

\item[Script mode] A way of using the Perl interpreter to read
code from a script and run it.
\index{script mode}

\item[one-liner mode:] A way of using the Perl interpreter to read
code passed at the operating system prompt and run it.
\index{one-liner mode}

\item[Script] A program stored in a file.
\index{script}

\item[Order of operations]  Rules governing the order in which
expressions involving multiple operators and operands are evaluated.
It is also called operator precedence.
\index{order of operations}
\index{operator precedence}
\index{precedence!operator}

\item[Concatenate]  To join two operands end-to-end.
\index{concatenation}

\item[Comment]  Information in a program that is meant for other
programmers (or anyone reading the source code) and has no effect on the
execution of the program.
\index{comment}

\item[Syntax error]  An error in a program that makes it impossible
to parse (and therefore impossible to compile and to run).
\index{syntax!error}

\item[Exception]  An error that is detected while the program is running.
\index{exception}

\item[Semantics]  The meaning of a program.
\index{semantics}

\item[Semantic error]   An error in a program that makes it do something
other than what the programmer intended.
\index{semantic error}

\end{description}


\section{Exercises}

\begin{exercise}

Repeating our advice from the previous chapter, whenever you learn
a new feature, you should try it out in interactive mode (under 
the REPL) and make errors on purpose to see what goes wrong.
\index{interactive mode}
\index{REPL}

\begin{itemize}

\item We've seen that {\tt \$n = 42} is legal.  What about {\tt 42 = \$n}?

\item How about {\tt \$x = \$y = 1}? (Hint: note that you will 
have to declare both variables, for example with a statement 
such as {\tt my \$x; my \$y;} or possibly {\tt my (\$x, \$y);}, 
before you can run the above.)

\item In some languages, statements don't have to end with a semi-colon, 
{\tt ;}. What happens in script mode if you omit a semi-colon at the end
of a Perl statement?
\index{semi-colon}

\item What if you put a period at the end of a statement?

\item In math notation you can multiply $x$ and $y$ like this: $x y$.
What happens if you try that in Perl?

\end{itemize}

\end{exercise}


\begin{exercise}

Practice using the Perl interpreter as a calculator: 
\index{calculator}

\begin{enumerate}

\item The volume of a sphere with radius $r$ is $\frac{4}{3} \pi r^3$.
  What is the volume of a sphere with radius 5?

\item Suppose the cover price of a book is \$24.95, but bookstores get a
  40\% discount.  Shipping costs \$3 for the first copy and 75 cents
  for each additional copy.  What is the total wholesale cost for
  60 copies?

\item If I leave my house at 6:52 a.m. and run 1 mile at an easy pace
  (8:15 per mile), then 3 miles at tempo (7:12 per mile) and 1 mile at
  easy pace again, what time do I get home for breakfast?
\index{running pace}

\end{enumerate}
\end{exercise}

