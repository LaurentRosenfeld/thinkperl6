% intro_part_1.tex -- Introduction to first part of the book
%

This book has been divided in two parts. The main reason for that
is that I wanted to make a distinction between on the one hand 
relatively basic notions that are really needed for any programmer 
using Perl 6, and on the other hand more advanced concepts that 
a good programmer needs to know but may be less often needed in 
the day-to-day development work.

The first eleven chapters (a bit more than 200~pages) which make up 
this first part are meant to teach the concepts that every 
programmer should know: variables, expressions, statements, 
functions, conditionals, recursion, operator precedence, loops, etc., 
as well as the basic data structures commonly used, and the most 
useful algorithms.  These chapters can, I believe, be the basis 
for a one-semester introductory course on programming. 

Of course, the professor or teacher that wishes to use this 
material is entirely free to skip some details from this 
Part~1 (and also to include sections from Part~2), but, at 
least, I have provided some guidelines on how I think this 
book could be used to teach programming using the Perl~6 language.

The second part focuses on different programming paradigms and 
more advanced programming techniques that are in my opinion of 
paramount importance, but should probably studied in the context 
of a second, more advanced, semester.

For now, let's get down to the basics. It is my hope that you 
will enjoy the trip.

